\subsection{Parte A1}%
\label{sub:cues_parte_a1}

\begin{enumerate}
	\item Sabiendo los conceptos de mediciones y medidas directas e indirectas.
		¿Cómo podríamos medir el trabajo experimental de la gravedad en la experiencia anterior
		(Primer laboratorio de Física I)\newline
		$W_g$ (Trabajo de la gravedad):\newline
		$A_c$ (Altura de caída libre)
		\begin{align*}
			W_g &= A_c * |\vec{A_e}| * cos(0)\\
			W_g &= 1.5m*|-9.98 \frac{m}{s^2} \widehat{\jmath}| \\
			W_g &= 14.97 J
		\end{align*}
	\item En dicha experiencia. ¿Cuáles son las variables a determinar experimentalmente?
		Y ¿Cuáles son las variables que se usan de manera teórica?
		\subitem La velocidad y la altura.
		\subitem La aceleración de la gravedad.
	\item ¿Por qué se propagan los errores?
		¿Cómo podríamos reducir el error obtenido en dicho experimento?
		\subitem Porque las herramientas son imperfectas.
		\subitem Usando herramientas más prescisas.
\end{enumerate}
