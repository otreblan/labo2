\subsection{Parte A3}%
\label{sub:cues_parte_a3}

\begin{enumerate}
	\item Graficar el DCL de los cuerpos
		(bloque, carrito y polea)
		definiendo el sistema de coordenadas de cada masa.
		%TODO DCL
	\item A partir de la medición realizada:
		\begin{itemize}
			\item Hacer gráficos de posición vs tiempo,
				velocidad vs tiempo y
				aceleración vs tiempo de la masa $m_2$.
				%TODO Esto está incompleto. Falta convertir los radianes a cm con respecto a la masa 2
				\begin{figure}[H]
					\centering
						\begin{tikzpicture}
							\begin{axis}[title=Posición vs tiempo]
								%\addplot[color=red,samples=100]{sin(deg(x))};
								%Esto no funcionaba por los finales de línea de microsoft
								\addplot[color=red]
									table [x=Tiempo(s) ,y=Angulo(rad) ,col sep=comma]
									{datos3.csv};
							\end{axis}
						\end{tikzpicture}
					\label{fig:pos_vs_tmp}
				\end{figure}
				\begin{figure}[H]
					\centering
						\begin{tikzpicture}
							\begin{axis}[title=Velocidad vs tiempo]
								%\addplot[color=red,samples=100]{sin(deg(x))};
								%Esto no funcionaba por los finales de línea de microsoft
								\addplot[color=red]
									table [x=Tiempo(s) ,y=Velocidad(rad/s) ,col sep=comma]
									{datos3.csv};
							\end{axis}
						\end{tikzpicture}
						\label{fig:vel_vs_tmp}
				\end{figure}
				\begin{figure}[H]
					\centering
						\begin{tikzpicture}
							\begin{axis}[title=Aceleración vs tiempo]
								%\addplot[color=red,samples=100]{sin(deg(x))};
								%Esto no funcionaba por los finales de línea de microsoft
								\addplot[color=red]
									table [x=Tiempo(s) ,y=Aceleracion(rad/s^2) ,col sep=comma]
									{datos3.csv};
							\end{axis}
						\end{tikzpicture}
					\label{fig:ace_vs_tmp}
				\end{figure}
			\item En el sistema,
				la aceleración con la que se mueve debe ser constante.
				¿Las \textbf{tres} curvas son las esperadas?
				¿Por qué sí?
				¿En todo el rango de datos?
			\item De acuerdo a los datos obtenidos,
				¿coincide el valor de la velocidad final medida con la calculada
				(usando relación trabajo-energía cinética)?
		\end{itemize}
\end{enumerate}
