\subsection{Parte A3}%
\label{sub:cues_parte_a3}

\begin{enumerate}
	\item Graficar el DCL de los cuerpos
		(bloque, carrito y polea)
		definiendo el sistema de coordenadas de cada masa.
		\begin{figure}[H]
			\centering
			\includesvg[width=0.4\textwidth]{DCL1.svg}
			\caption{Bloque}
			\label{fig:dcl_bloque}
		\end{figure}
		\begin{figure}[H]
			\centering
			\includesvg[width=0.4\textwidth]{DCL2.svg}
			\caption{Carrito}
			\label{fig:dcl_carro}
		\end{figure}
		\begin{figure}[H]
			\centering
			\includesvg[width=0.4\textwidth]{DCL3.svg}
			\caption{polea}
			\label{fig:dcl_polea}
		\end{figure}
	\item A partir de la medición realizada:
		\begin{itemize}
			\item Hacer gráficos de posición vs tiempo,
				velocidad vs tiempo y
				aceleración vs tiempo de la masa $m_2$.
				\begin{center}
					\huge{Hay que rehacer esta cosa}
				\end{center}
				\begin{figure}[H]
					\centering
						\begin{tikzpicture}
							\begin{axis}[
								title=\textbf{Posición vs tiempo},
								xlabel={tiempo ($s$)},
								ylabel={Posición ($cm$)},
								axis y line=left,
								axis x line=bottom,
								ymajorgrids=true,
								%xmin=0.5,
								%xmax=1.5,
								width=0.9\textwidth,
								height=8cm
								]
								%Esto no funcionaba por los finales de línea de microsoft
								\addplot[
									color=red,
									smooth,
									%restrict x to domain=0.5:1.4,
									draw opacity=0.1,
									ultra thick
									]
									table
									[
										x=Tiempo(s),
										%y expr=\thisrow{Angulo(rad)}*2.4,
										y expr=\thisrow{Angulo(rad)}*1.45,
										col sep=comma
									]
									{datos5.csv};
								\addplot[
									color=red,
									smooth,
									restrict x to domain=0.4:1.45,
									%draw opacity=0.1,
									ultra thick
									]
									table
									[
										x=Tiempo(s),
										%y expr=\thisrow{Angulo(rad)}*2.4,
										y expr=\thisrow{Angulo(rad)}*1.45,
										col sep=comma
									]
									{datos5.csv};
							\end{axis}
						\end{tikzpicture}
					\label{fig:pos_vs_tmp}
				\end{figure}
				\begin{figure}[H]
					\centering
						\begin{tikzpicture}
							\begin{axis}[
								title=\textbf{Velocidad vs tiempo},
								xlabel={tiempo ($s$)},
								ylabel={Velocidad ($ \frac{cm}{s} $)},
								axis y line=left,
								axis x line=bottom,
								ymajorgrids=true,
								%xmin=0.5,
								%xmax=1.5,
								width=0.9\textwidth,
								height=8cm
								]
								%Esto no funcionaba por los finales de línea de microsoft
								\addplot[
									color=red,
									smooth,
									draw opacity=0.1,
									ultra thick
									]
									table
									[
										x=Tiempo(s),
										%y expr=\thisrow{Velocidad(rad/s)}*2.4,
										y expr=\thisrow{Velocidad(rad/s)}*1.45,
										col sep=comma
									]
									{datos5.csv};
								\addplot[
									color=red,
									smooth,
									restrict x to domain=0.4:1.45,
									%draw opacity=0.1,
									ultra thick
									]
									table
									[
										x=Tiempo(s),
										%y expr=\thisrow{Velocidad(rad/s)}*2.4,
										y expr=\thisrow{Velocidad(rad/s)}*1.45,
										col sep=comma
									]
									{datos5.csv};
							\end{axis}
						\end{tikzpicture}
						\label{fig:vel_vs_tmp}
				\end{figure}
				\begin{figure}[H]
					\centering
						\begin{tikzpicture}
							[
							spy using outlines={rectangle,
												magnification=2,
												drop shadow,
												width=8cm,
												height=3cm,
												connect spies
												}
							]
							\begin{axis}[
								title=\textbf{Aceleración vs tiempo},
								xlabel={tiempo ($s$)},
								ylabel={Aceleración ($ \frac{cm}{s^2} $)},
								axis y line=left,
								axis x line=bottom,
								ymajorgrids=true,
								axis line style = ultra thick,
								%xmin=0.5,
								%xmax=1.5,
								width=0.9\textwidth,
								height=8cm
								]
								%Esto no funcionaba por los finales de línea de microsoft
								\addplot[
									color=red,
									smooth,
									draw opacity=0.1,
									ultra thick
									]
									table
									[
										x=Tiempo(s),
										%y expr=\thisrow{Aceleracion(rad/s^2)}*2.4,
										y expr=\thisrow{Aceleracion(rad/s^2)}*1.45,
										col sep=comma
									]
									{datos5.csv};
								\addplot[
									color=red,
									smooth,
									restrict x to domain=0.4:1.45,
									%draw opacity=0.1,
									ultra thick
									]
									table
									[
										x=Tiempo(s),
										%y expr=\thisrow{Aceleracion(rad/s^2)}*2.4,
										y expr=\thisrow{Aceleracion(rad/s^2)}*1.45,
										col sep=comma
									]
									{datos5.csv};
								\coordinate (ripple) at (0.95,-50);
								\coordinate (lupa) at (4,400);
							\end{axis}
							\spy[black] on (ripple) in node[fill=white] at (lupa);
						\end{tikzpicture}
					\label{fig:ace_vs_tmp}
				\end{figure}
			\item En el sistema,
				la aceleración con la que se mueve debe ser constante.
				¿Las \textbf{tres} curvas son las esperadas?
				¿Por qué sí?
				¿Por qué no?
				¿En todo el rango de datos?
				%\subitem No.
				%\subitem Porque el carrito tenía un resorte y habían fuerzas despreciadas.
				%\subitem Desde el primer choque del del resorte
			\item De acuerdo a los datos obtenidos,
				¿coincide el valor de la velocidad final medida con la calculada
				(usando relación trabajo-energía cinética)?
				\subitem No. \textbf{Hay que volver a hacer las ecuaciones}
				%Si la velocidad final esta en el momento del primer choque.
				%\begin{align*}
					%W &= 0.3m*|m_2g* \frac{24mm}{14.5mm}| *\cos(0)\\
					%W &= 0.3m*|0.100502kg*9.81 \frac{m}{s^2} * \frac{24mm}{14.5mm}| \\
					%W &= 1.63J = K \\
					%K &= \frac{m_1v_1^2}{2} \\
					%v_1 &= \sqrt{ \frac{0.49J*2}{1.0344kg} }\\
					%v_1 &= 0.97 \frac{m}{s} &&\text{En el gráfico $|v_1|> 0.1 \frac{m}{s} $}
				%\end{align*}
		\end{itemize}
\end{enumerate}
