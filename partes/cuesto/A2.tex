\subsection{Parte A2}%
\label{sub:cues_parte_a2}

\begin{enumerate}
	\item Al medir el valor de $\pi$
		¿Se trata de una medición directa o indirecta?
		¿En cuál forma de medición se tiene mayor $\% error$?
		¿A qué se debe esto?
	\item ¿Cómo pueden definir a $1\ rad$?
		¿Cuál es la relación con grados sexagesimales y de dónde proviene esa relación?
		¿Es lo mismo usar grados sexagesimales y radianes?
		Si no es lo mismo
		¿En qué casos se debe usar una u otra unidad de medida?
		\subitem Al arco de un círculo de ángulo $1\ rad$ y radio 1.
		\subitem
		\begin{align*}
			\frac{rad}{sxag} &= \frac{2\pi}{360^\circ} \\
		\end{align*}
		\begin{figure}[H]
			\centering
			\includesvg[width=0.4\textwidth]{radGrad.svg}
		\end{figure}
		\subitem No
		\subitem Radianes cuando se necesita saber algo relacionado con el perímetro.
		\subitem Y grados cuando no tienes una calculadora.
	\item En el siguente esquema:
		¿Cómo se relacionan la rapidez del bloque 1 con el bloque 2?
		Sí se conociera la rapidez del bloque 2
		¿Cómo se calcularía la velocidad angular de la polea?
		Mostrar el procedimiento de cálculo de cada pregunta.
		\begin{figure}[H]
			\centering
			\includesvg[width=0.4\textwidth]{esquema.svg}
		\end{figure}
\end{enumerate}
